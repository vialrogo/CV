\documentclass[a4paper,10pt]{article}

\usepackage[utf8]{inputenc}
\usepackage[brazil]{babel}
\usepackage{marvosym}
\usepackage{fontspec}
\usepackage{xunicode,xltxtra,url,parskip}
\RequirePackage{color,graphicx}
\usepackage[usenames,dvipsnames]{xcolor}
\usepackage[big]{layaureo}
\usepackage{titlesec}
\usepackage{hyperref}

% Links
\definecolor{linkcolour}{rgb}{0,0.2,0.6}
\hypersetup{colorlinks,breaklinks,urlcolor=linkcolour, linkcolor=linkcolour}

% Fontes
\defaultfontfeatures{Mapping=tex-text}
\setmainfont[SmallCapsFont = Fontin SmallCaps]{Fontin}

% Secções
\titleformat{\section}{\Large\scshape\raggedright}{}{0em}{}[\titlerule]
\titlespacing{\section}{0pt}{3pt}{3pt}

% Documento
\begin{document}
\pagestyle{empty} % Páginas não numeradas

% _________________________________________ Título _________________________________________
\par{\centering {\Huge \textsc{Victor Alberto Romero}}\bigskip\par}

% _____________________________________ Dados Pessoais _____________________________________
\section{Dados Pessoais}

\begin{tabular}{rl}
  \textsc{Local e data de nascimento} & Cali, Colômbia  | 26 Julho 1988 \\
  \textsc{Endereço:}                  & Avenida Jaguaré, 297. Bloco Andrea, apartamento 83 \\
  \textsc{Telefone:}                  & +55 (11) 95906~-~0632 \\
  \textsc{Email:}                     & \href{mailto:vialrogo@gmail.com}{vialrogo@gmail.com} \\
\end{tabular}

% _________________________________ Experiencia de Mercado _________________________________
\section{Experiência de Mercado}
\begin{tabular}{rl}

  \textsc{Set 2012~-~Jan 2013}  & Professor na \textsc{Universidad del Valle}, Colômbia \\
                                &\footnotesize{Professor da disciplina Fundamentos de Linguagens de Programação.} \\
                                &\\

  \textsc{Ago~-~Dez 2012}       & Professor na \textsc{Universidad del Valle}, Colômbia \\
                                &\footnotesize{Professor da disciplina Arquitetura de Computadores I.} \\
                                &\\

  \textsc{Fev 2012~-~Jan 2013}  & Engenheiro de Suporte em \textsc{Centro de Estudios Brasileros}, Colômbia \\
                                &\footnotesize{Administração de software e hardware. Suporte a contabilidade.} \\

\end{tabular}

% ________________________________________ Estágios ________________________________________
\section{Estágios}
\begin{tabular}{rl}

  \textsc{Nov 2015~-~Hoje}      & Pesquisador na \textsc{Universidade de São Paulo}, Brasil \\
                                &\footnotesize{Laboratório de Dinâmica e Instrumentação} \\

  \textsc{Jun 2014~-~Dez 2014}  & Monitor na \textsc{Universidade de São Paulo}, Brasil \\
                                &\footnotesize{Monitor da disciplina Introdução à Computação} \\

  \textsc{Jun 2010~-~Dez 2010}  & Monitor na \textsc{Universidad del Valle}, Colômbia \\
                                &\footnotesize{Monitor da disciplina Sistemas de Informação.} \\

  \textsc{Fev 2010~-~Jun 2010}  & Monitor na \textsc{Universidad del Valle}, Colômbia \\
                                &\footnotesize{Monitor da disciplina Fundamentos de Linguagens de Programação.} \\

  \textsc{Jun 2009~-~Dez 2009}  & Monitor na \textsc{Universidad del Valle}, Colômbia \\
                                &\footnotesize{Monitor da disciplina Sistemas de Informação.} \\

  \textsc{Jun 2008~-~Dez 2008}  & Pesquisador na \textsc{Universidad del Valle}, Colômbia \\
                                &\footnotesize{Manual para criação de aplicações usando SmartCards} \\

  \textsc{Fev 2008~-~Jun 2008}  & Monitor na \textsc{Universidad del Valle}, Colômbia \\
                                &\footnotesize{Monitor da disciplina Introdução à Programação Orientada a Objetos} \\

\end{tabular}
% ________________________________________ Educação ________________________________________
\section{Educação}
\begin{tabular}{rl}	

  \textsc{Ago 2014} & \textsc{Mestrado em Ciência da Computação}, Universidade de São Paulo, Brasil \\
                    & Incompleto | Ênfase: Otimização \\
                    & Dissertação: ``\textsc{Uso de programação não-linear para a resolução do problema}\\
                    & \textsc{de empacotamento de círculos aninhados}'' \\
                    & Orientador: Prof.~Dr.~Ernesto Birgin \\
                    &\\

  \textsc{Ago 2012} & \textsc{Graduação em Engenharia da Computação}, Universidad del Valle, Colômbia \\
                    & Ênfase: Redes de Comunicações \\
                    & Dissertação: ``\textsc{Análisis del flujo de datos en redes de mediante teoría del caos}'' \\
                    & Orientador: Prof.~Dr.~Angel García Baños \\
                    &\\

  \textsc{Ago 2010} & \textsc{Graduação em Engenharia Eletrônica}, Universidad del Valle, Colômbia \\
                    & Ênfase: Redes de Comunicações \\
                    & Dissertação: ``\textsc{Diseño e implementación de una herramienta software para la} \\
                    & \textsc{supervisión y registro de llamadas en centralitas telefónicas}'' \\
                    & Orientador: Prof.~Leandro Villa, Msc. \\
                    &\\

\end{tabular}

% ____________________________________ Cursos Adicionais ___________________________________
\section{Cursos Adicionais}
\begin{tabular}{rl}	

  \textsc{Mai 2011} & \textsc{Instalação e Administração de servidores HP EVA 4400 y HP Blade.} \\
                    & Universidad del Valle, Colômbia \\
                    & Curso curto: 24 horas \\

\end{tabular}
% ______________________________________ Certificados ______________________________________
\section{Distinções}
\begin{tabular}{rl}

  \textsc{Jun 2011} & Estímulos por alto rendimento acadêmico \footnotesize(Melhor média de notas: 4.36 de 5)\normalsize\\
  \textsc{Jun 2010} & Estímulos por alto rendimento acadêmico \footnotesize(Melhor média de notas: 4.41 de 5)\normalsize\\
  \textsc{Dez 2005} & Estímulos por alto rendimento acadêmico \footnotesize(Terceira melhor média de notas: 4.42 de 5)\normalsize\\
  \textsc{Jun 2005} & Estímulos por alto rendimento acadêmico \footnotesize(Terceira melhor média de notas: 4.55 de 5)\normalsize\\
  \textsc{Dez 2004} & Estímulos por alto rendimento acadêmico \footnotesize(Segunda melhor média de notas: 4.65 de 5)\normalsize\\

\end{tabular}

% _________________________________________ Idiomas ________________________________________
\section{Idiomas}
\begin{tabular}{rll}

  \textsc{Espanhol:}    & Língua Nativa     & \\
  \textsc{Português:}   & Fluente           & Prova \textsc{CELPE-BRAS} (INEP) 2011: Nível Intermediário Superior\\
  \textsc{Inglês:}      & Intermediário     & Prova \textsc{MET} (Michigan Institute) 2012: Nível B2 \\

\end{tabular}

% ___________________________________ Competências Técnicas ________________________________
\section{Competências Técnicas}
\begin{tabular}{rl}

  Conhecimentos Básicos:        & Matlab, PHP, JavaScript, MySQL, PostgreSQL, OracleDB \\
  Conhecimentos Intermediários: & Java, C, C++, Qt, Python, Bash, Otimização Não-linear, Linux\\
  Conhecimentos Avançados:      & \LaTeX, Computação Evolutiva 

\end{tabular}

% _________________________________ Interesses e Atividades ________________________________
\section{Interesses e Atividades}
Tecnologia, Software Livre, Programação, Educação, Pesquisa \\
Administração de Servidores Linux e Automatização de processos \\
Domótica e Eletrônica de baixo custo \\
Viagens, Culinária

% ________________________________________ Footnote ________________________________________
\vfill
\hrulefill\\
\center{\footnotesize{\today}}

\end{document}
