\documentclass[a4paper,10pt]{article}

\usepackage[utf8]{inputenc}
\usepackage[brazil]{babel}
\usepackage[top=2.5cm, bottom=2.5cm, left=2.5cm, right=2.5cm]{geometry}
\usepackage{marvosym}
\usepackage{fontspec}
\usepackage{xunicode,xltxtra,url,parskip}
\RequirePackage{color,graphicx}
\usepackage[usenames,dvipsnames]{xcolor}
\usepackage{titlesec}
\usepackage{hyperref}
\usepackage{longtable}
\usepackage{booktabs}

% Links
\definecolor{linkcolour}{rgb}{0,0.2,0.6}
\hypersetup{colorlinks,breaklinks,urlcolor=linkcolour, linkcolor=linkcolour}

% Fontes
\defaultfontfeatures{Mapping=tex-text}
\setmainfont[SmallCapsFont = Fontin SmallCaps]{Fontin}

% Secções
\titleformat{\section}{\Large\scshape\raggedright}{}{0em}{}[\titlerule]
\titlespacing{\section}{0pt}{3pt}{3pt}

% Documento
\begin{document}
\pagestyle{empty} % Páginas não numeradas

% _________________________________________ Título _________________________________________
\par{\centering {\Huge \textsc{Victor Alberto Romero}}\bigskip\par}

% _____________________________________ Dados Pessoais _____________________________________
\section{Dados Pessoais}

\begin{longtable}{rl}
  \textsc{Local e data de nascimento}   & Cali, Colômbia  | 26 Julho 1988 \\
  \textsc{Endereço:}                    & Avenida Jaguaré, 297. Bloco Andrea, apartamento 83\\
                                        & São Paulo, Brasil\\
  \textsc{Telefone:}                    & +55 (11) 95906 0632 \\
  \textsc{Email:}                       & \href{mailto:vialrogo@gmail.com}{vialrogo@gmail.com} \\
                                        & \href{mailto:vialrogo@usp.br}{vialrogo@usp.br} \\
\end{longtable}

% _________________________________ Projetos Finalizados ___________________________________
\section{Projetos Relevantes Finalizados}

\begin{longtable}{rl}
    \textsc{Ago 2016~-~Dez 2016}    & Criação de um sistema de calibração automático para hidrofones em \\
                                    & banda de frequência de 3kHz a 200kHz \\
                                    &\footnotesize{Resumo: O projeto foi desenvolvido em duas etapas. A primeira etapa foi a adequação}\\
                                    &\footnotesize{de um tanque de $27m^3$. A segunda etapa foi a montagem da instrumentação para a}\\
                                    &\footnotesize{projeção e recepção do sinal. Foram usados como hidrofones de referência}\\
                                    &\footnotesize{hidrofones Brüel \& Kjær. Todo o Software foi desenvolvido em Matlab.} \\
\end{longtable}

\begin{longtable}{rl}
    \textsc{Mai 2016~-~Nov 2016}    & Monitoramento de eventos acústicos (dragagem) do porto de Sepetiba\\
                                    &\footnotesize{Resumo: O objetivo desse projeto foi analisar o impacto acústico durante o processo}\\
                                    &\footnotesize{de dragagem na baía de Sepetiba. Foram realizadas gravações das pressões sonoras antes}\\
                                    &\footnotesize{e durante o processo de dragagem. Após a coleta dos dados, os mesmos foram processados}\\ 
                                    &\footnotesize{e analisados através de SPLs e espectrogramas. Todos os softwares de análise foram }\\
                                    &\footnotesize{desenvolvidos em Matlab e Shell Script.} \\
\end{longtable}

\begin{longtable}{rl}
    \textsc{Set 2016~-~Fev 2017}    & Monitoramento acústico terrestre da floresta estadual Águas de Santa \\
                                    & Bárbara, São Paulo,  Brasil \\
                                    &\footnotesize{Resumo: O objetivo é o monitoramento acústico contínuo de 4 biomas do cerrado brasileiro:}\\ 
                                    &\footnotesize{Cerrado típico, cerrado de campo, banhadão e cerradão. Foram desenvolvidos e instalados}\\ 
                                    &\footnotesize{4 esquipamentos para o monitoramento, com alimentação via bateria e energia solar. Foram}\\
                                    &\footnotesize{utilizados sistemas embarcados e sistemas microprocessados. Os dados foram coletados}\\ 
                                    &\footnotesize{e gerou-se espectrogramas diários de cada bioma.}\\
\end{longtable}

\begin{longtable}{rl}
    \textsc{Out 2016~-~Ago 2017}    & Desenvolvimento de um sistema de calibração automático de microfones e \\
                                    & hidrofones no ar\\ 
                                    &\footnotesize{Resumo: O objetivo é a adaptação do sistema de calibração na água para o ar, visando}\\ 
                                    &\footnotesize{contornar as limitações dos hidrofones nesse meio. Criou-se um sistema que permite}\\ 
                                    &\footnotesize{gerar as curvas de respostas para baixas frequências (20Hz – 20KHz), assim como um}\\ 
                                    &\footnotesize{sistema de calibração em unidades absolutas a través de comparação.}\\
\end{longtable}

% _________________________________ Projetos Em Andamento  _________________________________
\section{Projetos Relevantes em Andamento}
\begin{longtable}{rl}
    \textsc{Jun 2016~-~Atual}   & Paisagens acústicas submarinas no litoral de São Paulo\\
                                &\footnotesize{Auxílio Pesquisa FAPESP no. 2016/02175--0.}\\
                                &\footnotesize{O projeto visa o monitoramento de longa duração da paisagem acústica submarina em duas}\\ 
                                &\footnotesize{Unidades de Conservação (UC) Marinhas no centro do litoral de São Paulo; Parque Estadual}\\  
                                &\footnotesize{Marinho da Laje de Santos e Estação Ecológica Tupinambás. O monitoramento será realizado}\\ 
                                &\footnotesize{tilizando um equipamento de monitoramento de acústica submarina autônomo, desenvolvido}\\ 
                                &\footnotesize{pelo próprio laboratório, e chamado de OceanPod. A análise do banco de dados permitirá}\\ 
                                &\footnotesize{bter, pioneiramente, um conhecimento sobre a paisagem acústica submarina no litoral de São}\\
                                &\footnotesize{aulo, a variação dos níveis de pressão sonora ao longo do tempo, a eventual correção}\\ 
                                &\footnotesize{spacial entre as UCS, a identificação e estudo das principais fontes sonoras presentes,}\\ 
                                &\footnotesize{lém de fornecer importantes informações sobre impactos antrópicos para a gestão das}\\ 
                                &\footnotesize{nidades de Conservação.}\\
\end{longtable}

\begin{longtable}{rl}
    \textsc{Jan 2017~-~Atual}   & Criação de um sistema de detecção automático de botos\\
                                &\footnotesize{Resumo: O objetivo é a criação de um sistema que permita a detecção de eventos acústicos} \\ 
                                &\footnotesize{específicos (assobios de botos) a partir de gravações de áudio de longa duração.} \\
\end{longtable}

% _________________________________ Projetos Em Andamento  _________________________________
\section{Atividades Relevantes Recentes}

\begin{itemize}
    \item Calibração de hidrofones em baixa frequência (50Hz – 3kHz);
    \item Calibração de hidrofones em alta frequência usando sistema de calibração própria (3KHz – 200KHz);
    \item Calibração de microfones em banda de frequência audível (50Hz – 20KHz);
    \item Levantamento de curva de resposta em frequência para amplificadores e pré-amplificadores;
    \item Desenvolvimento e montagem de equipamento específicos para gravações acústicas terrestre e marinhas de longa duração (Sistema de bateria e energia Solar);
    \item Trabalho em campo: Instalação e coleta de equipamentos de monitoramento acústico de longa duração, assim como sua manutenção;
    \item Manipulação e processamento de grandes bancos de dados acústicos.\\
\end{itemize}

% _________________________________ Experiência de Mercado _________________________________
\section{Experiência Profissional}
\begin{longtable}{rl}

    \textsc{Ago 2016~-~Atual}       & Pesquisador na \textsc{Universidade de São Paulo~-~USP}, Brasil \\
                                    &\footnotesize{Estudante de mestrado e pesquisador do laboratório LACMAM} \\
                                    &\\

    \textsc{Abr 2017~-~Mai 2017}    & Pesquisador em modelagem acúStica submarina, Brasil \\
                                    &\footnotesize{Apoio em pesquisa para o Instituto de Pesquisas da Marinha (IPqM)} \\
                                    &\footnotesize{em atividades de eletrônica analógica e digital relacionadas ao} \\
                                    &\footnotesize{desenvolvimento do gravador acústico submarino} \\
                                    &\\

    \textsc{Set 2012~-~Jan 2013}    & Professor na \textsc{Universidad del Valle}, Colômbia \\
                                    &\footnotesize{Professor da disciplina Fundamentos de Linguagens de Programação.} \\
                                    &\\

    \textsc{Ago 2012~-~Dez 2012}    & Professor na \textsc{Universidad del Valle}, Colômbia \\
                                    &\footnotesize{Professor da disciplina Arquitetura de Computadores I.} \\
                                    &\\

    \textsc{Fev 2012~-~Jan 2013}    & Engenheiro de Suporte em \textsc{Centro de Estudios Brasileros}, Colômbia \\
                                    &\footnotesize{Administração de software e hardware. Suporte a contabilidade.} \\
\end{longtable}

% ________________________________________ Estágios ________________________________________
\section{Estágios}
\begin{longtable}{rl}

    \textsc{Nov 2015~-~Ago 2016}  & Pesquisador na \textsc{Universidade de São Paulo}, Brasil \\
                                  &\footnotesize{Laboratório de Acústica e Meio Ambiente} \\
                                  &\\

    \textsc{Jun 2014~-~Dez 2014}  & Monitor na \textsc{Universidade de São Paulo}, Brasil \\
                                  &\footnotesize{Monitor da disciplina Introdução à Computação} \\
                                  &\\

    \textsc{Jun 2010~-~Dez 2010}  & Monitor na \textsc{Universidad del Valle}, Colômbia \\
                                  &\footnotesize{Monitor da disciplina Sistemas de Informação.} \\
                                  &\\

    \textsc{Fev 2010~-~Jun 2010}  & Monitor na \textsc{Universidad del Valle}, Colômbia \\
                                  &\footnotesize{Monitor da disciplina Fundamentos de Linguagens de Programação.} \\
                                  &\\

    \textsc{Jun 2009~-~Dez 2009}  & Monitor na \textsc{Universidad del Valle}, Colômbia \\
                                  &\footnotesize{Monitor da disciplina Sistemas de Informação.} \\
                                  &\\

    \textsc{Jun 2008~-~Dez 2008}  & Pesquisador na \textsc{Universidad del Valle}, Colômbia \\
                                  &\footnotesize{Manual para criação de aplicações usando SmartCards} \\
                                  &\\

    \textsc{Fev 2008~-~Jun 2008}  & Monitor na \textsc{Universidad del Valle}, Colômbia \\
                                  &\footnotesize{Monitor da disciplina Introdução à Programação Orientada a Objetos} \\

\end{longtable}
% ________________________________________ Educação ________________________________________
\section{Educação}
\begin{longtable}{rl}	
    \textsc{Ago 2016~-~Atual}       & Mestrado em Engenharia Mecânica\\
                                    & Universidade de São Paulo, Brasil \\
                                    & Ênfase: Processamento de Sinais \\
                                    & Dissertação: ``\textsc{Reconhecimento de padrões em análise de paisagens acústicas}\\
                                    & \textsc{de alta banda de frequência}'' \\
                                    & Orientador: Prof.~Dr.~Linilson R. Padovese \\
                                    &\\

    \textsc{Fev 2013~-~Ago 2014}    & Mestrado em Ciência da Computação\\
                                    & Universidade de São Paulo, Brasil \\
                                    & Incompleto | Ênfase: Otimização \\
                                    & Dissertação: ``\textsc{Uso de programação não-linear para a resolução}\\
                                    & \textsc{do problema de empacotamento de círculos aninhados}'' \\
                                    & Orientador: Prof.~Dr.~Ernesto Birgin \\
                                    &\\

    \textsc{Ago 2006~-~Ago 2012}    & Graduação em Engenharia da Computação\\
                                    & Universidad del Valle, Colômbia \\
                                    & Ênfase: Inteligência artificial \\
                                    & Dissertação: ``\textsc{Análisis del flujo de datos en redes de comunicaciones}\\
                                    & \textsc{mediante teoría del caos}'' \\
                                    & Orientador: Prof.~Dr.~Angel García Baños \\
                                    &\\

    \textsc{Ago 2004~-~Ago 2010}    & Graduação em Engenharia Eletrônica\\
                                    & Universidad del Valle, Colômbia \\
                                    & Ênfase: Redes de Comunicações \\
                                    & Dissertação: ``\textsc{Diseño e implementación de una herramienta software} \\
                                    & \textsc{para la supervisión y registro de llamadas en centralitas telefónicas}'' \\
                                    & Orientador: Prof.~Leandro Villa, Msc. \\
\end{longtable}

% _______________________________________ Distinções _______________________________________
\section{Distinções}

\begin{longtable}{rl}	
    \textsc{Jun 2011} & \textsc{Estímulos por alto rendimento acadêmico (Melhor média de notas: 4.36 de 5)} \\
    \textsc{Jun 2010} & \textsc{Estímulos por alto rendimento acadêmico (Melhor média de notas: 4.41 de 5)} \\
    \textsc{Dez 2005} & \textsc{Estímulos por alto rendimento acadêmico (Melhor média de notas: 4.42 de 5)} \\
    \textsc{Jun 2005} & \textsc{Estímulos por alto rendimento acadêmico (Melhor média de notas: 4.55 de 5)} \\
    \textsc{Dez 2004} & \textsc{Estímulos por alto rendimento acadêmico (Melhor média de notas: 4.65 de 5)} \\
\end{longtable}

% ____________________________________ Cursos Adicionais ___________________________________
\section{Cursos Adicionais}

\begin{longtable}{rl}	
  \textsc{Mai 2011} & \textsc{Instalação e Administração de servidores HP EVA 4400 y HP Blade.} \\
                    & Universidad del Valle, Colômbia \\
                    & Carga horária: 24 horas \\
\end{longtable}

% _________________________________________ Idiomas ________________________________________
\section{Idiomas}

\begin{longtable}{rl}
    \textsc{Espanhol:}      & Língua Nativa\\
                            &\\
    \textsc{Português:}     & Fluente\\
                            & Prova \textsc{CELPE-BRAS} (INEP) 2011: Nível Intermediário Superior\\
                            &\\
    \textsc{Inglês:}        & Intermediário\\
                            & Prova \textsc{MET} (Michigan Institute) 2012: Nível B2 \\
\end{longtable}

% ___________________________________ Competências Técnicas ________________________________
\section{Competências Técnicas}
\begin{longtable}{rl}

    Conhecimentos Avançados:        & Matlab, Python, C, C++, Qt, Bash, \LaTeX, Linux\\
                                    & Machine Learning, Raspberry PI\\
                                    &\\

    Conhecimentos Intermediários:   & R, Java, PHP, JavaScript, Otimização Não-linear, SQL\\
                                    &\\

\end{longtable}

% _________________________________ Interesses e Atividades ________________________________
\section{Outros Interesses}

\begin{itemize}
    \item Tecnologia de consumo
    \item Software Livre
    \item Educação
    \item Administração de Servidores Linux
    \item Automatização de processos
    \item Eletrônica de baixo custo
    \item Viagens
\end{itemize}



% ________________________________________ Footnote ________________________________________
\vfill
\hrulefill\\
\center{\footnotesize{\today}}

\end{document}
